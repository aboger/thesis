
\section{Examples}

Some examples\dots

\subsection{A labeled subsection}
\label{ssec:label}

This is Section \ref{ssec:label} on page \pageref{ssec:label}.

\subsubsection{A subsubsection}

\dots

\subsection{Lists}

\begin{enumerate}
\item Some
\item fancy
\item items!
\end{enumerate}

\begin{itemize}
\item Some
\item fancy
\item items!
\end{itemize}

\subsection{Tables}

A table.

\begin{center} 
\begin{minipage}{\linewidth}
	\centering 
	\begin{tabular}{ *4{C{.185\textwidth}}}\toprule[1.5pt]
		\bf  \# & \bf Some & \bf fancy & \bf titles\\\midrule
		1 & 5 & 3,582 & 100\% \\
		2 & Hello & \dots & 50\%\\
		\bottomrule[1.25pt]
	\end {tabular}
	\par
	\captionof{table}{A table.}
	\label{tab:table} 
\end{minipage}
\end{center}


\subsubsection{Acronyms along the example \acs{http}}

First call prints \ac{http}, second call prints \ac{http}. Use \acs{http} in titles.

\subsection{A todo box}

\todo{Buy me a beer!}

\subsection{Citations}

A simple citation~\cite{rfc791}. Citation with page~\cites[p.~12]{rfc2068}. A couple of references~\cites[p.~12;]{rfc2068}{rfc791}.

\citebox{A citation box\dots}{rfc2068}

I recommend to manage your references using the JabRef tool.

After new citations, run: PDFLatex, BibTex, PDFLatex, PDFLatex!

\subsection{Source code}

A Java code listing. Current default language is Java. This can be changed in \textit{settings.tex}.

\lstinputlisting[
	caption={The \code{Hello} class.},
	label={lst:Hello}
]{code/Hello.java}

\subsection{TikZ figures}

\begin{itemize}
\item Write your figures in a standalone document. See \textit{tikz/example.tex}.
\item Compile it.
\item Include the output PDF file.
\end{itemize}

This enables embedding for your figures into non-\LaTeX{} environments such as a web site. It is possible to include the code of the standalone document into this document, too. Therefore, call \verb#\#\verb#input{tikz/example.tex}#. However, I don't recommend to do this, since this may cause problems due to page width differences and several other reasons.

\begin{figure}[ht]
	\centering
	
	\includegraphics{tikz/example.pdf}
        
    \caption{A fancy TikZ figure.}
    \label{fig5:projectstack}
\end{figure}

